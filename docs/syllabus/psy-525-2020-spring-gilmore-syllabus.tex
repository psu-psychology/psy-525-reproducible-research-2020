% Options for packages loaded elsewhere
\PassOptionsToPackage{unicode}{hyperref}
\PassOptionsToPackage{hyphens}{url}
%
\documentclass[
]{article}
\usepackage{lmodern}
\usepackage{amssymb,amsmath}
\usepackage{ifxetex,ifluatex}
\ifnum 0\ifxetex 1\fi\ifluatex 1\fi=0 % if pdftex
  \usepackage[T1]{fontenc}
  \usepackage[utf8]{inputenc}
  \usepackage{textcomp} % provide euro and other symbols
\else % if luatex or xetex
  \usepackage{unicode-math}
  \defaultfontfeatures{Scale=MatchLowercase}
  \defaultfontfeatures[\rmfamily]{Ligatures=TeX,Scale=1}
\fi
% Use upquote if available, for straight quotes in verbatim environments
\IfFileExists{upquote.sty}{\usepackage{upquote}}{}
\IfFileExists{microtype.sty}{% use microtype if available
  \usepackage[]{microtype}
  \UseMicrotypeSet[protrusion]{basicmath} % disable protrusion for tt fonts
}{}
\makeatletter
\@ifundefined{KOMAClassName}{% if non-KOMA class
  \IfFileExists{parskip.sty}{%
    \usepackage{parskip}
  }{% else
    \setlength{\parindent}{0pt}
    \setlength{\parskip}{6pt plus 2pt minus 1pt}}
}{% if KOMA class
  \KOMAoptions{parskip=half}}
\makeatother
\usepackage{xcolor}
\IfFileExists{xurl.sty}{\usepackage{xurl}}{} % add URL line breaks if available
\IfFileExists{bookmark.sty}{\usepackage{bookmark}}{\usepackage{hyperref}}
\hypersetup{
  pdftitle={PSY 525.001 Spring 2020 Syllabus},
  hidelinks,
  pdfcreator={LaTeX via pandoc}}
\urlstyle{same} % disable monospaced font for URLs
\usepackage[margin=1in]{geometry}
\usepackage{longtable,booktabs}
% Correct order of tables after \paragraph or \subparagraph
\usepackage{etoolbox}
\makeatletter
\patchcmd\longtable{\par}{\if@noskipsec\mbox{}\fi\par}{}{}
\makeatother
% Allow footnotes in longtable head/foot
\IfFileExists{footnotehyper.sty}{\usepackage{footnotehyper}}{\usepackage{footnote}}
\makesavenoteenv{longtable}
\usepackage{graphicx}
\makeatletter
\def\maxwidth{\ifdim\Gin@nat@width>\linewidth\linewidth\else\Gin@nat@width\fi}
\def\maxheight{\ifdim\Gin@nat@height>\textheight\textheight\else\Gin@nat@height\fi}
\makeatother
% Scale images if necessary, so that they will not overflow the page
% margins by default, and it is still possible to overwrite the defaults
% using explicit options in \includegraphics[width, height, ...]{}
\setkeys{Gin}{width=\maxwidth,height=\maxheight,keepaspectratio}
% Set default figure placement to htbp
\makeatletter
\def\fps@figure{htbp}
\makeatother
\usepackage[normalem]{ulem}
% Avoid problems with \sout in headers with hyperref
\pdfstringdefDisableCommands{\renewcommand{\sout}{}}
\setlength{\emergencystretch}{3em} % prevent overfull lines
\providecommand{\tightlist}{%
  \setlength{\itemsep}{0pt}\setlength{\parskip}{0pt}}
\setcounter{secnumdepth}{-\maxdimen} % remove section numbering

\title{PSY 525.001 Spring 2020 Syllabus}
\author{}
\date{\vspace{-2.5em}}

\begin{document}
\maketitle

\hypertarget{transparent-open-and-reproducible-research-practices-in-the-social-and-behavioral-sciences}{%
\subsection{Transparent, Open, and Reproducible Research Practices in
the Social and Behavioral
Sciences}\label{transparent-open-and-reproducible-research-practices-in-the-social-and-behavioral-sciences}}

Sidney Harris, New Yorker Magazine

\hypertarget{instructor}{%
\subsection{Instructor}\label{instructor}}

Rick O. Gilmore, Ph.D. Professor of Psychology 114 Moore Building

+1 (814) 865-3664 rogilmore AT psu DOT edu
\href{http://doodle.com/rickgilmore}{Schedule an appointment}

\url{https://www.personal.psu.edu/rog1}
\url{https://gilmore-lab.github.io} \url{https://databrary.org}

\hypertarget{meeting-location-and-time}{%
\subsection{Meeting Location and Time}\label{meeting-location-and-time}}

Please note that beginning Tuesday, March 17 and continuing until
further notice the class will meet on Zoom at
\url{https://psu.zoom.us/my/rogilmore} in order to help reduce the
impact of the COVID-19 outbreak.

Tues 09:05-11:50 pm, \sout{350
Moore}\url{https://psu.zoom.us/my/rogilmore} January 6 - May 1, 2020
course 17294

\hypertarget{resources}{%
\subsection{Resources}\label{resources}}

\begin{itemize}
\tightlist
\item
  Awesome, free-to-you, data science tutorials from .
\item
  Psychology Department
  \href{https://psu-psychology.github.io/data-science-and-reproducibility/}{page}
  and
  \href{https://psu-psychology.github.io/data-science-and-reproducibility/forum.html}{discussion
  group} about data science and reproducibility.
\end{itemize}

\begin{center}\rule{0.5\linewidth}{0.5pt}\end{center}

\hypertarget{about-the-course}{%
\subsection{About the course}\label{about-the-course}}

Is there a crisis of reproducibility in psychological science? What does
it mean to ask the question? What are transparent, open, and
reproducible research practices? Should one implement them? How? This
course will seek answers to these questions. Students will read about
recent failures in research transparency and reproducibility and discuss
the ethics of open practices in scientific research. Through a series of
guided exercises students will learn how to use new research tools
(e.g.~\href{https://www.rstudio.com}{RStudio},
\href{http://jupyter.org}{Jupyter/iPython}, the
\href{http://osf.io}{Open Science Framework},
\href{http://github.com}{GitHub},
\href{https://en.wikipedia.org/wiki/Command-line_interface}{command line
scripting}) that make it easier to implement open and reproducible
research practices. At the end of the course, students will be capable
of implementing one or more new research practices into their own
workflows. Evaluation will be based on in-class participation, short
papers, and project assignments. No prior software development
experience is required, but a willingness to learn new skills is
essential.

\begin{center}\rule{0.5\linewidth}{0.5pt}\end{center}

\hypertarget{schedule}{%
\section{Schedule}\label{schedule}}

\hypertarget{week-1-tuesday-january-14-2020}{%
\subsection{Week 1: Tuesday, January 14,
2020}\label{week-1-tuesday-january-14-2020}}

\hypertarget{topics}{%
\subsubsection{Topics}\label{topics}}

\begin{itemize}
\tightlist
\item
  Introduction to the course
\item
  Needs/interests assessment, and goal setting
\item
  Discussion: Why trust science?
\item
  Setting up computational environments

  \begin{itemize}
  \tightlist
  \item
    On your local machine
  \item
    In the cloud
  \end{itemize}
\end{itemize}

\hypertarget{homework}{%
\subsubsection{Homework}\label{homework}}

Due by the start of class on 2020-01-21.

\begin{enumerate}
\def\labelenumi{\arabic{enumi}.}
\tightlist
\item
  In a paragraph or two, discuss whether you think researchers in your
  field \emph{do} and \emph{should} embrace ``Mertonian norms.'' Why or
  why not?
\item
  In a paragraph or two, describe your current knowledge of computer
  programming languages, and at least three learning goals you have for
  building upon that base.
\item
  Create a GitHub account, send me your account name in an email, and
  see if you can create your own copy (fork or clone) of the course
  repository. If you succeed, take a screen shot of the repository in
  your local GitHub account.
\item
  Write these items up in a Word (.docx) file, and email it to me. Make
  sure to use a sensible file name, e.g.,
  \texttt{psy525-YOUR\_LAST\_NAME-2020-01-21.docx}.
\end{enumerate}

\hypertarget{week-2-tuesday-january-21-2020}{%
\subsection{Week 2: Tuesday, January 21,
2020}\label{week-2-tuesday-january-21-2020}}

\hypertarget{topics-1}{%
\subsubsection{Topics}\label{topics-1}}

\begin{itemize}
\tightlist
\item
  The values of science
\item
  Cases of scientific misconduct
\item
  What is reproducibility? Are we in a crisis?
\end{itemize}

\hypertarget{readingswebinars}{%
\subsubsection{Readings/webinars}\label{readingswebinars}}

\begin{itemize}
\tightlist
\item
  The values of science (read 1; skim the others)

  \begin{itemize}
  \tightlist
  \item
    Nosek, B. A., \& Bar-Anan, Y. (2012). Scientific utopia I: Opening
    scientific communication. \emph{Psychological Inquiry},
    \emph{23}(3), 217--243. Retrieved May 9, 2015, from
    \url{http://dx.doi.org/10.1080/1047840X.2012.692215}
  \item
    Kim, S. Y., \& Kim, Y. (2018). The ethos of science and its
    correlates: An empirical analysis of scientists' endorsement of
    Mertonian norms. \emph{Science, Technology and Society}, 23(1),
    1--24. SAGE Publications India. Retrieved from
    \url{https://doi.org/10.1177/0971721817744438}
  \item
    Brakewood, B., \& Poldrack, R. A. (2013). The ethics of secondary
    data analysis: Considering the application of Belmont principles to
    the sharing of neuroimaging data. \emph{NeuroImage}, \emph{82},
    671--676. Retrieved from
    \url{http://dx.doi.org/10.1016/j.neuroimage.2013.02.040}
  \end{itemize}
\item
  Cases of scientific misconduct (read 1; skim the other)

  \begin{itemize}
  \tightlist
  \item
    \href{http://www.sciencemag.org/news/2012/09/harvard-psychology-researcher-committed-fraud-us-investigation-concludes}{Hauser}
  \item
    \href{http://www.nytimes.com/2013/04/28/magazine/diederik-stapels-audacious-academic-fraud.html?pagewanted=all\&_r=0}{Stapel}
  \end{itemize}
\item
  Reproducibility (skim)

  \begin{itemize}
  \tightlist
  \item
    Open Science Collaboration. (2015). Estimating the reproducibility
    of psychological science. \emph{Science}, \emph{349}(6251),
    aac4716--aac4716. \url{https://doi.org/10.1126/science.aac4716}
  \end{itemize}
\item
  Supplemental (not required)

  \begin{itemize}
  \tightlist
  \item
    Heesen, R., \& Bright, L. K. (2019). Is peer review a good idea?
    \emph{The British Journal for the Philosophy of Science}, \emph{40}.
    eprints.lse.ac.uk. Retrieved January 7, 2020, from
    \url{http://eprints.lse.ac.uk/101242/}
  \item
    Goodman, S. N., Fanelli, D., \& Ioannidis, J. P. A. (2016). What
    does research reproducibility mean? \emph{Science Translational
    Medicine}, 8(341), 341ps12--341ps12.
    \url{https://doi.org/10.1126/scitranslmed.aaf5027}
  \item
    \url{http://www.stats.org/what-do-we-mean-by-reproducibility/}
  \end{itemize}
\end{itemize}

\hypertarget{homework-1}{%
\subsubsection{Homework}\label{homework-1}}

Due by the start of class on 2020-01-28.

\begin{enumerate}
\def\labelenumi{\arabic{enumi}.}
\tightlist
\item
  Choose one of the verification bias examples from
  \href{https://pure.mpg.de/pubman/item/item_1569964_8/component/file_1569966/Stapel_Investigation_Final_report.pdf}{\emph{Flawed
  Science}}. In a paragraph or two, propose ways you might avoid this
  sort of bias in your own research.
\item
  Choose one of the \href{https://osf.io/9f6gx/}{TOP guideline
  categories} where either your own research practices have room to
  improve or you are doing rather well. In a paragraph or two, explain
  your reasoning.
\end{enumerate}

\hypertarget{week-3-tuesday-january-28-2020}{%
\subsection{Week 3: Tuesday, January 28,
2020}\label{week-3-tuesday-january-28-2020}}

\hypertarget{topics-2}{%
\subsubsection{Topics}\label{topics-2}}

\begin{itemize}
\tightlist
\item
  Evaluating Munafò et al.'s (2017) ``Manifesto for reproducible
  science'', \url{https://doi.org/10.1038/s41562-016-0021}
\item
  Workflows and \emph{methods} reproducibility
\item
  Tidy data
\end{itemize}

\hypertarget{readingswebinars-1}{%
\subsubsection{Readings/webinars}\label{readingswebinars-1}}

\begin{itemize}
\tightlist
\item
  Munafò, M. R., Nosek, B. A., Bishop, D. V. M., Button, K. S.,
  Chambers, C. D., Sert, N. P. du, \ldots{} Ioannidis, J. P. A. (2017).
  A manifesto for reproducible science. Nature Human Behaviour, 1, 0021.
  \url{https://doi.org/10.1038/s41562-016-0021}. (\textbf{Required})
\item
  Reproducible workflows (\textbf{Optional})

  \begin{itemize}
  \tightlist
  \item
    \url{http://datasci.kitzes.com/lessons/python/reproducible_workflow.html}
  \item
    \url{http://grunwaldlab.github.io/Reproducible-science-in-R/index.html}
  \end{itemize}
\item
  Tidy data (\textbf{Optional})

  \begin{itemize}
  \tightlist
  \item
    Wickham, H. (2014). Tidy Data. Journal of Statistical Software,
    59(10). \url{http://dx.doi.org/10.18637/jss.v059.i10}.
  \end{itemize}
\item
  File naming (\textbf{Optional})

  \begin{itemize}
  \tightlist
  \item
    \url{http://guides.nyu.edu/data_management/file_org}
  \item
    \url{http://kbroman.org/dataorg/}
  \end{itemize}
\item
  Spreadsheets (\textbf{Optional})

  \begin{itemize}
  \tightlist
  \item
    \url{https://github.com/jennybc/sanesheets}
  \item
    \url{http://www.datacarpentry.org/2015-03-09-ISI-CODATA/lessons/excel/ecology-examples/00-intro.html}
  \item
    \url{https://github.com/jennybc/scary-excel-stories}
  \end{itemize}
\end{itemize}

\hypertarget{homework-2}{%
\subsubsection{Homework}\label{homework-2}}

Due by the start of class on 2020-01-28.

\begin{enumerate}
\def\labelenumi{\arabic{enumi}.}
\tightlist
\item
  \sout{Pick one of the recommended elements from
  \href{http://www.nature.com/articles/s41562-016-0021/tables/1}{Table 1
  in Munafò, et al.~(2017)}. Evaluate the recommendation. Do you agree
  that it would mitigate one or more threats to reproducibility. Why or
  why not? Do you agree with the assessment about the degree to which
  stakeholders have adopted the recommended practice?}. We covered most
  of this material in class.
\item
  Edit/create a text-based workflow for an active project you are
  working on.

  \begin{itemize}
  \tightlist
  \item
    Annote the workflow to indicate where it could be made more
    reproducible, transparent.
  \item
    If you are feeling super-ambitious, you may want to try creating a
    graph-based workflow using the \texttt{diagrammeR} package.
  \end{itemize}
\end{enumerate}

\hypertarget{week-4-tuesday-february-4-2020}{%
\subsection{Week 4: Tuesday, February 4,
2020}\label{week-4-tuesday-february-4-2020}}

\hypertarget{topics-3}{%
\subsubsection{Topics}\label{topics-3}}

\begin{itemize}
\tightlist
\item
  Pregistration and registered reports
\item
  Introduction to RStudio and R Markdown
\end{itemize}

\hypertarget{readingswebinars-2}{%
\subsubsection{Readings/webinars}\label{readingswebinars-2}}

\begin{itemize}
\tightlist
\item
  Pre-registration and registered reports

  \begin{itemize}
  \tightlist
  \item
    Chambers, C., Munafò, M., \& signatories, more than 80. (2013, June
    5). Trust in science would be improved by study pre-registration.
    \emph{The Guardian}. Retrieved from
    \url{https://www.theguardian.com/science/blog/2013/jun/05/trust-in-science-study-pre-registration}
  \item
    Registered Reports. (n.d.). Retrieved January 24, 2017, from
    \url{https://cos.io/rr/?_ga=1.163722943.1251838540.1458403228}
  \item
    Mathot, S. (2013, March 26). The pros and cons of pre-registration
    in fundamental research. Retrieved January 24, 2017 from
    \url{http://www.cogsci.nl/blog/miscellaneous/215-the-pros-and-cons-of-pre-registration-in-fundamental-research}
  \item
    (Optional) Frank, M. (2016, July 22). Preregister everything.
    \url{http://babieslearninglanguage.blogspot.com/2016/07/preregister-everything.html}
  \item
    (Optional) Claesen, A., Gomes, S. L. B. T., Tuerlinckx, F., \&
    Vanpaemel, W. (2019, May). Preregistration: Comparing Dream to
    Reality. Retrieved from psyarxiv.com/d8wex.
  \end{itemize}
\item
  \href{http://rmarkdown.rstudio.com/}{R Markdown} exercise

  \begin{itemize}
  \tightlist
  \item
    \href{https://www.rstudio.com/resources/webinars/getting-started-with-r-markdown/}{Getting
    started with R Markdown}
  \end{itemize}
\item
  Optional

  \begin{itemize}
  \tightlist
  \item
    \href{https://www.datacamp.com/courses/reporting-with-r-markdown}{Authoring
    R Markdown Reports}. 3 hrs.
  \item
    \href{https://www.datacamp.com/courses/working-with-the-rstudio-ide-part-1}{Working
    with the RStudio IDE (Part II)}. 3 hrs.
  \item
    \href{https://www.datacamp.com/courses/working-with-the-rstudio-ide-part-2}{Working
    with the RStudio IDE (Part II)}
  \end{itemize}
\end{itemize}

\hypertarget{homework-3}{%
\subsubsection{Homework}\label{homework-3}}

Due by the start of class on 2020-02-11.

\begin{enumerate}
\def\labelenumi{\arabic{enumi}.}
\tightlist
\item
  Find a preregistration document for a study relevant to your research
  interests on \texttt{aspredicted.org} or \texttt{osf.io}. In a few
  paragraphs, comment on what was and what was not included. Would the
  preregistration provide researchers sufficient structure to carry out
  the research without `HARKing'? What downsides do you see to
  preregistration?
\item
  Create a template for a reproducible research report or a
  pregistration document in R Markdown.
\item
  Convert some portion of your workflow from last week's assignment into
  an R Markdown.
\item
  Make sure to use some R Markdown features you haven't used before.
  Flag the features you used and explain how they'd be useful in an
  actual use case you can envision.
\item
  Either email me the R Markdown file, or if you've pushed your file to
  a GitHub repo, send me a review request.
\end{enumerate}

\hypertarget{week-5-tuesday-february-11-2020}{%
\subsection{Week 5: Tuesday, February 11,
2020}\label{week-5-tuesday-february-11-2020}}

\hypertarget{topics-4}{%
\subsubsection{Topics}\label{topics-4}}

\begin{itemize}
\tightlist
\item
  Version control
\item
  git

  \begin{itemize}
  \tightlist
  \item
    GitHub
  \item
    GitHub pages
  \end{itemize}
\end{itemize}

\hypertarget{readingswebinars-3}{%
\subsubsection{Readings/webinars}\label{readingswebinars-3}}

\begin{itemize}
\tightlist
\item
  Hardwicke, T. E., \& Ioannidis, J. P. A. (2018). Mapping the universe
  of registered reports. \emph{Nature Human Behaviour}, \emph{2}(11),
  793--796. nature.com. Retrieved from
  \url{http://dx.doi.org/10.1038/s41562-018-0444-y}
\item
  Claesen, A., Gomes, S. L. B. T., Tuerlinckx, F., \& Vanpaemel, W.
  (2019, May). Preregistration: Comparing dream to reality. Retrieved
  from \url{https://psyarxiv.com/d8wex}.
\item
  \href{https://www.rstudio.com/resources/webinars/rstudio-essentials-webinar-series-managing-part-2/}{GitHub
  and RStudio}
\item
  Jenny Bryan's \url{http://happygitwithr.com/}
\end{itemize}

\hypertarget{homework-4}{%
\subsubsection{Homework}\label{homework-4}}

Due by the start of class on 2020-02-18.

\begin{enumerate}
\def\labelenumi{\arabic{enumi}.}
\tightlist
\item
  Create GitHub repo for the project you completed last week

  \begin{itemize}
  \tightlist
  \item
    Open an issue flagging \texttt{@rogilmore} so I know to look at your
    repo and document.
  \end{itemize}
\item
  Create a repo for your final course project

  \begin{itemize}
  \tightlist
  \item
    Create an R Markdown document where you start to outline the
    possible directions that your final project might take.
  \item
    Open an issue so I can take a look.
  \end{itemize}
\item
  Clone a repo; fix/change something; make a
  \href{https://help.github.com/articles/about-pull-requests/}{pull
  request}.

  \begin{itemize}
  \tightlist
  \item
    Option 1:
    \url{http://psu-psychology.github.io/psy-525-reproducible-research-2020/}

    \begin{itemize}
    \tightlist
    \item
      Suggestion: Add something about yourself to \texttt{students.html}
      by editing \texttt{students.Rmd} and then rebuilding the site via
      \texttt{rmarkdown::render\_site(encoding\ =\ "utf8")}
    \end{itemize}
  \item
    Option 2:
    \url{https://psu-psychology.github.io/data-science-and-reproducibility/}

    \begin{itemize}
    \tightlist
    \item
      Suggestion: Add or edit \texttt{resources.html} by editing
      \texttt{resources.Rmd}.
    \end{itemize}
  \end{itemize}
\end{enumerate}

\hypertarget{week-6-tuesday-february-18-2020}{%
\subsection{Week 6: Tuesday, February 18,
2020}\label{week-6-tuesday-february-18-2020}}

\hypertarget{topics-5}{%
\subsubsection{Topics}\label{topics-5}}

\begin{itemize}
\tightlist
\item
  Simulation as a tool for reproducible and transparent science
\item
  Visualization tools in R
\end{itemize}

\hypertarget{readingswebinars-4}{%
\subsubsection{Readings/webinars}\label{readingswebinars-4}}

\begin{itemize}
\tightlist
\item
  Required.
  \href{https://www.datacamp.com/courses/data-visualization-with-ggplot2-1}{Data
  Visualization with ggplot2 (Part 1)}
\item
  Desirable.
  \href{https://www.datacamp.com/courses/data-visualization-with-ggplot2-2}{Data
  Visualization with ggplot2 (Part 2)}
\end{itemize}

\hypertarget{homework-5}{%
\subsubsection{Homework}\label{homework-5}}

Due by the start of class on 2020-02-25.

\begin{enumerate}
\def\labelenumi{\arabic{enumi}.}
\tightlist
\item
  Create your own simulated data set for a real or proposed study.

  \begin{itemize}
  \tightlist
  \item
    You may adapt or build upon the examples used in class.
  \item
    Put the results in an R Markdown (.Rmd) file.
  \item
    Commit the .Rmd file to your private repo on GitHub and either raise
    an Issue on GitHub or submit a pull request.
  \end{itemize}
\item
  Plot the results of your simulation using ggplot2 commands.
\item
  Make sure that your simulation has the following sub-sections:

  \begin{itemize}
  \tightlist
  \item
    Introduction/Motivation

    \begin{itemize}
    \tightlist
    \item
      What are you simulating and why? Where do the parameter estimates
      come from? The literature or your best guess?
    \end{itemize}
  \item
    Plots
  \item
    Statistical Analyses
  \item
    Discussion/Conclusions

    \begin{itemize}
    \tightlist
    \item
      What did you discover or demonstrate?
    \end{itemize}
  \end{itemize}
\end{enumerate}

\begin{itemize}
\tightlist
\item
  Please label the R chunks in your R Markdown files
\end{itemize}

\hypertarget{week-7-tuesday-february-25-2020}{%
\subsection{Week 7: Tuesday, February 25,
2020}\label{week-7-tuesday-february-25-2020}}

\hypertarget{topics-6}{%
\subsubsection{Topics}\label{topics-6}}

\begin{itemize}
\tightlist
\item
  Doing other useful things with R and R Markdown
\end{itemize}

\hypertarget{readingswebinars-5}{%
\subsubsection{Readings/webinars}\label{readingswebinars-5}}

\hypertarget{homework-6}{%
\subsubsection{Homework}\label{homework-6}}

Due by the start of class on 2020-03-03.

\begin{enumerate}
\def\labelenumi{\arabic{enumi}.}
\tightlist
\item
  Create a set of HTML talk slides using R Markdown.
\item
  Try rendering the slides as a Word document, PowerPoint document, or
  PDF.
\item
  Create a new repo and generate a simple website using R Markdown.
\end{enumerate}

\hypertarget{week-8-tuesday-march-3-2020}{%
\subsection{Week 8: Tuesday, March 3,
2020}\label{week-8-tuesday-march-3-2020}}

\hypertarget{topics-7}{%
\subsubsection{Topics}\label{topics-7}}

\begin{itemize}
\tightlist
\item
  Interactive visualizations using \href{shiny.html}{Shiny apps}
\end{itemize}

\hypertarget{readingswebinars-recommended}{%
\subsubsection{Readings/webinars
(recommended)}\label{readingswebinars-recommended}}

\begin{itemize}
\tightlist
\item
  \href{https://www.rstudio.com/resources/webinars/how-to-start-with-shiny-part-1/}{How
  to start with Shiny (Part 1)}
\item
  \href{https://www.rstudio.com/resources/webinars/how-to-start-with-shiny-part-2/}{How
  to start with Shiny (Part 2)}
\item
  \href{https://www.rstudio.com/resources/webinars/how-to-start-with-shiny-part-3/}{How
  to start with Shiny (Part 3)}
\end{itemize}

\hypertarget{homework-7}{%
\subsubsection{Homework}\label{homework-7}}

Due by the start of class on 2020-03-17.

\begin{enumerate}
\def\labelenumi{\arabic{enumi}.}
\tightlist
\item
  Complete a 1-2 page write-up describing your plans for your final
  project.
\end{enumerate}

\hypertarget{spring-break-march-9---12-2020}{%
\subsection{Spring Break March 9 - 12,
2020}\label{spring-break-march-9---12-2020}}

\textbf{NO CLASS}

\hypertarget{week-9-tuesday-march-17-2020}{%
\subsection{Week 9: Tuesday, March 17,
2020}\label{week-9-tuesday-march-17-2020}}

\hypertarget{topics-8}{%
\subsubsection{Topics}\label{topics-8}}

\begin{itemize}
\tightlist
\item
  Python, the other language of data science

  \begin{itemize}
  \tightlist
  \item
    Intro to \href{http://jupyter.org/}{Jupyter}
  \item
    Jupyter for research, teaching, talks
  \end{itemize}
\end{itemize}

\hypertarget{readingswebinars-6}{%
\subsubsection{Readings/webinars}\label{readingswebinars-6}}

\begin{itemize}
\tightlist
\item
  \href{https://www.datacamp.com/courses/importing-data-in-python-part-1}{Importing
  data in Python (Part 1)}
\item
  \href{https://www.datacamp.com/courses/importing-data-in-python-part-2}{Importing
  data in Python (Part 2)}
\end{itemize}

\hypertarget{homework-8}{%
\subsubsection{Homework}\label{homework-8}}

Due by the start of class on 2020-03-24.

TBD

\hypertarget{week-10-tuesday-march-24-2020}{%
\subsection{Week 10: Tuesday, March 24,
2020}\label{week-10-tuesday-march-24-2020}}

\hypertarget{topics-9}{%
\subsubsection{Topics}\label{topics-9}}

\begin{itemize}
\tightlist
\item
  Tools for reproducible data-gathering

  \begin{itemize}
  \tightlist
  \item
    E-Prime, Matlab (\href{http://psychtoolbox.org/}{Psychophysics
    Toolbox}), \href{http://www.psychopy.org/}{PsychoPy}.
  \item
    \href{http://www.jspsych.org/}{jsPsych}
  \item
    MTurk

    \begin{itemize}
    \tightlist
    \item
      \href{https://psiturk.org/}{psiTurk}
    \end{itemize}
  \end{itemize}
\end{itemize}

\hypertarget{readingswebinars-7}{%
\subsubsection{Readings/webinars}\label{readingswebinars-7}}

\begin{itemize}
\tightlist
\item
  de Leeuw, J.R. (2015). jsPsych: A JavaScript library for creating
  behavioral experiments in a Web browser. \emph{Behavior Research
  Methods}, \emph{47}(1), 1-12. \url{doi:10.3758/s13428-014-0458-y}
\end{itemize}

\hypertarget{homework-9}{%
\subsubsection{Homework}\label{homework-9}}

Due by the start of class on 2020-03-31.

\begin{enumerate}
\def\labelenumi{\arabic{enumi}.}
\tightlist
\item
  Open a PsychoPy demo program and save it with a new name. Add
  additional documentation to the demo program that explains what's
  happening. Change one or more parameters to make the program do
  something slightly different, and explain what parameters you changed.
\end{enumerate}

\hypertarget{week-11-tuesday-march-31-2020}{%
\subsection{Week 11: Tuesday, March 31,
2020}\label{week-11-tuesday-march-31-2020}}

\hypertarget{topics-10}{%
\subsubsection{Topics}\label{topics-10}}

\begin{itemize}
\tightlist
\item
  Using APIs

  \begin{itemize}
  \tightlist
  \item
    U.S. Census
  \item
    Google Drive
  \item
    Box
  \item
    Wikidata
  \item
    Databrary
  \item
    OSF
  \end{itemize}
\end{itemize}

\hypertarget{homework-10}{%
\subsubsection{Homework}\label{homework-10}}

Due by the start of class on 2020-04-07.

\begin{enumerate}
\def\labelenumi{\arabic{enumi}.}
\tightlist
\item
  Create a Jupyter notebook where you document your exploration of the
  U.S. Census, Box, or Google Drive APIs.
\end{enumerate}

\hypertarget{week-12-tuesday-april-7-2020}{%
\subsection{Week 12: Tuesday, April 7,
2020}\label{week-12-tuesday-april-7-2020}}

\hypertarget{topics-11}{%
\subsubsection{Topics}\label{topics-11}}

\begin{itemize}
\tightlist
\item
  Where to share?
\item
  Publishing data
\item
  Challenges to sharing
\item
  Your open science portfolio
\item
  Funder policies
\end{itemize}

\hypertarget{readingswebinars-8}{%
\subsubsection{Readings/webinars}\label{readingswebinars-8}}

\begin{itemize}
\tightlist
\item
  Meyer, M. N. (2018). Practical Tips for Ethical Data Sharing. Advances
  in Methods and Practices in Psychological Science, 2515245917747656.
  SAGE Publications Inc.~Retrieved from
  \url{https://doi.org/10.1177/2515245917747656}
\item
  Gilmore, R.O. et al.~(2020).
\end{itemize}

\hypertarget{homework-11}{%
\subsubsection{Homework}\label{homework-11}}

Due by the start of class on 2020-04-07.

\begin{enumerate}
\def\labelenumi{\arabic{enumi}.}
\tightlist
\item
  Choose two outlets for sharing research data or materials and compare
  and contrast their strengths and weaknesses.
\end{enumerate}

\hypertarget{week-13-tuesday-april-14-2020}{%
\subsection{Week 13: Tuesday, April 14,
2020}\label{week-13-tuesday-april-14-2020}}

\hypertarget{topics-12}{%
\subsubsection{Topics}\label{topics-12}}

\begin{itemize}
\tightlist
\item
  Catch-up week
\end{itemize}

\hypertarget{week-14-tuesday-april-21-2020}{%
\subsection{Week 14: Tuesday, April 21,
2020}\label{week-14-tuesday-april-21-2020}}

\hypertarget{topics-13}{%
\subsubsection{Topics}\label{topics-13}}

\begin{itemize}
\tightlist
\item
  Preparation for student projects
\end{itemize}

\hypertarget{week-15-tuesday-april-28-2020}{%
\subsection{Week 15: Tuesday, April 28,
2020}\label{week-15-tuesday-april-28-2020}}

\hypertarget{topics-14}{%
\subsubsection{Topics}\label{topics-14}}

\begin{itemize}
\tightlist
\item
  Student project presentations
\end{itemize}

\begin{center}\rule{0.5\linewidth}{0.5pt}\end{center}

\hypertarget{evaluation}{%
\section{Evaluation}\label{evaluation}}

PSY 525 course performance will be evaluated based on the following
scheme:

\begin{longtable}[]{@{}lll@{}}
\toprule
Component & Points & \% of Grade\tabularnewline
\midrule
\endhead
Class participation & 2 pts/class * 15 weeks = 30 & 25\tabularnewline
Assignments & 5 pts * 12 assignments = 60 & 50\tabularnewline
Final projects & 30 & 25\tabularnewline
\textbf{TOTAL} & \textbf{150} & \textbf{100}\tabularnewline
\bottomrule
\end{longtable}

\begin{center}\rule{0.5\linewidth}{0.5pt}\end{center}

\hypertarget{resources-1}{%
\section{Resources}\label{resources-1}}

\hypertarget{how-tos}{%
\subsection{How-to's}\label{how-tos}}

\begin{itemize}
\tightlist
\item
  How to \href{how_to/rstudio-server-pro.html}{connect to Penn State's
  RStudio Server Pro} instance to run RStudio in the cloud.
\item
  How to \href{how_to/jupyter-notebooks-psu.html}{connect to Penn
  State's Jupyter notebook server}.
\item
  How to \href{how_to/google-colaboratory.html}{use Google's
  colaboratory Jupyter notebook service}.
\end{itemize}

\hypertarget{web-courses}{%
\subsection{Web courses}\label{web-courses}}

\begin{itemize}
\tightlist
\item
  \href{https://rstudio.com/conference/}{RStudio.conf}

  \begin{itemize}
  \tightlist
  \item
    \href{https://github.com/rstudio/rstudio-conf}{GitHub repo for 2020
    conference}
  \end{itemize}
\item
  \href{https://datacarpentry.org/}{Data Carpentry}, now
  \href{https://carpentries.org/}{The Carpentries}

  \begin{itemize}
  \tightlist
  \item
    \href{https://datacarpentry.org/openrefine-socialsci/}{OpenRefine
    for Social Science Data} by Data Carpentry.
  \item
    \href{http://swcarpentry.github.io/r-novice-gapminder/}{R for
    Reproducible Scientific Analysis}
  \item
    \href{http://swcarpentry.github.io/python-novice-inflammation/}{Programming
    with Python}
  \item
    \href{http://swcarpentry.github.io/python-novice-gapminder/}{Plotting
    and Programming in Python}
  \item
    \href{http://swcarpentry.github.io/git-novice/}{Version Control with
    Git}
  \end{itemize}
\end{itemize}

\hypertarget{books-and-e-books}{%
\subsection{Books and E-Books}\label{books-and-e-books}}

\begin{itemize}
\tightlist
\item
  Chang, W. (2012). \emph{R Graphics Cookbook}.
  \url{http://ase.tufts.edu/bugs/guide/assets/R\%20Graphics\%20Cookbook.pdf}
\item
  Gandrud, C. (2015). \emph{Reproducible Research with R and R Studio,
  Second Edition}. CRC Press.
  \url{https://www.crcpress.com/Reproducible-Research-with-R-and-R-Studio-Second-Edition/Gandrud/p/book/9781498715379}
\item
  VanderPlas, J. (2016). \emph{Python Data Science Handbook}.
  \url{https://jakevdp.github.io/PythonDataScienceHandbook/}
\item
  Wickham, H. \& Grolemund, G. (2016). \emph{R for Data Science: Import,
  Tidy, Transform, Visualize, and Model Data}. O'Reilly Media.
  \url{http://shop.oreilly.com/product/0636920034407.do}.
\item
  Wickham, H. (2019). \emph{Advanced R, 2nd Ed}. Chapman \& Hall/CRC.
  \url{https://adv-r.hadley.nz/}.
\item
  Xie, Y., Allaire, J.J., \& Grolemund, G. (2019). \emph{R Markdown: The
  Definitive Guide}. Chapman \& Hall/CRC.
  \url{https://bookdown.org/yihui/rmarkdown/}.
\end{itemize}

\hypertarget{blogsprojects}{%
\subsection{Blogs/Projects}\label{blogsprojects}}

\begin{itemize}
\tightlist
\item
  R bloggers \url{https://www.r-bloggers.com/}
\item
  Open science in practice.
  \url{https://inattentionalcoffee.wordpress.com/2017/01/03/open-science-in-practice/}
\item
  Practicing what I preach: The open Ph.D.~experiment.
  \url{http://www.librarianinthecity.com/2016/12/practicing-what-i-preach-the-open-phd-experiment/}
\item
  \href{https://sites.trinity.edu/osl}{Open Stats Lab}. Use data
  published with psychology papers to teach statistics.
\item
  Jupyter, Zeppelin, Beaker: The Rise of the Notebooks.
  \url{https://opendatascience.com/blog/jupyter-zeppelin-beaker-the-rise-of-the-notebooks/}
\item
  Open science and free culture.
  \url{http://www.simoncolumbus.com/publications-2/open-science-and-free-culture/}
\item
  \href{https://osf.io/wfc6u/}{Collaborative Replications and Education
  Project (CREP)}.
\end{itemize}

\hypertarget{software-tools}{%
\subsection{Software tools}\label{software-tools}}

\begin{itemize}
\tightlist
\item
  OpenRefine, \url{https://openrefine.org}
\item
  LaTeX on the web

  \begin{itemize}
  \tightlist
  \item
    ShareLaTex \url{http://sharelatex.com}
  \item
    Overleaf \url{http://www.overleaf.com}
  \item
    Authorea \url{http://www.authorea.com}
  \end{itemize}
\item
  Experimental Design Assistant for planning animal studies
  \url{https://eda.nc3rs.org.uk/}
\item
  p-hacker Shiny app \url{http://shinyapps.org/apps/p-hacker/}
\end{itemize}

\hypertarget{articles}{%
\subsection{Articles}\label{articles}}

\begin{itemize}
\tightlist
\item
  Schweinsberg, M., Madan, N., Vianello, M., Sommer, S. A., Jordan, J.,
  Tierney, W., \ldots{} Uhlmann, E. L. (2016). The pipeline project:
  Pre-publication independent replications of a single laboratory's
  research pipeline. Journal of Experimental Social Psychology, 66,
  55--67. \url{https://doi.org/10.1016/j.jesp.2015.10.001}
\item
  Munafò, M. R., Nosek, B. A., Bishop, D. V. M., Button, K. S.,
  Chambers, C. D., Sert, N. P. du, \ldots{} Ioannidis, J. P. A. (2017).
  A manifesto for reproducible science. Nature Human Behaviour, 1, 0021.
  \url{https://doi.org/10.1038/s41562-016-0021}.
\item
  Poldrack, R. A., Baker, C. I., Durnez, J., Gorgolewski, K. J.,
  Matthews, P. M., Munafò, M. R., \ldots{} Yarkoni, T. (2017). Scanning
  the horizon: towards transparent and reproducible neuroimaging
  research. Nature Reviews Neuroscience, advance online publication.
  \url{https://doi.org/10.1038/nrn.2016.167}.
\end{itemize}

\hypertarget{citation-management}{%
\subsection{Citation Management}\label{citation-management}}

\hypertarget{comparisons}{%
\subsubsection{Comparisons}\label{comparisons}}

\begin{itemize}
\item
  \url{http://guides.lib.uchicago.edu/c.php?g=297307\&p=1984557}
\item
  \url{https://en.wikipedia.org/wiki/Comparison_of_reference_management_software}
\end{itemize}

\hypertarget{software}{%
\subsubsection{Software}\label{software}}

\begin{itemize}
\tightlist
\item
  Mendeley
  \url{https://www.mendeley.com/careers/?dgcid=google_paid-search_Mendeley.com-Search-Branded-USA\&gclid=Cj0KEQiAifvEBRCVx5up6Ojgr5oBEiQALHw1TviV-SG0qOSpmmSkoIbrYrn1ZvPeZTinianXeiQa6FIaAqA28P8HAQ}
\item
  Zotero \url{https://www.zotero.org/}
\item
  Paperpile \url{https://paperpile.com/}
\end{itemize}

\end{document}
